% !TEX TS-program = xelatex
% !TEX encoding = UTF-8 Unicode
% !Mode:: "TeX:UTF-8"

\documentclass{resume}
\usepackage{zh_CN-Adobefonts_external} % Simplified Chinese Support using external fonts (./fonts/zh_CN-Adobe/)
% \usepackage{NotoSansSC_external}
% \usepackage{NotoSerifCJKsc_external}
% \usepackage{zh_CN-Adobefonts_internal} % Simplified Chinese Support using system fonts
\usepackage{linespacing_fix} % disable extra space before next section
\usepackage{cite}

\begin{document}
\pagenumbering{gobble} % suppress displaying page number

\name{桂博文}

\basicInfo{
  \email{bwgui203@gmail.com} \textperiodcentered\ 
  \phone{(+86) 139-1775-6429} \textperiodcentered\
  \age{34} \textperiodcentered\
 
\section{\faGraduationCap\  教育背景}
\datedsubsection{\textbf{上海交通大学}, 上海}{2013 -- 2016}
\textit{硕士研究生}\ 自动化
\datedsubsection{\textbf{上海交通大学}, 上海}{2008 -- 2012}
\textit{学士}\ 热能与动力工程

\section{\faUsers\ 职业背景}
\datedsubsection{\textbf{蚂蚁集团}}{2018.6 -- 至今}
\textit{技术专家(P7)}
\datedsubsection{\textbf{中金所技术公司}}{2016.7 -- 2018.6}
\textit{C++开发工程师}

\section{\faUsers\ 重点项目经历}
\datedsubsection{\textbf{OceanBase}}{2022.3 -- 至今}
负载均衡transfer功能开发
\begin{itemize}
  \item 实现了 xxx 特性
  \item 后台资源占用率减少8\%
  \item xxx
\end{itemize}

\datedsubsection{\textbf{分布式科学上网姿势}}{2014年6月 -- 至今}
\role{Golang, Linux}{个人项目,和富帅糕合作开发}
\begin{onehalfspacing}
分布式负载均衡科学上网姿势, https://github.com/cyfdecyf/cow
\begin{itemize}
  \item 修复了连接未正常关闭导致文件描述符耗尽的 bug
  \item 使用Chord 哈希 URL, 实现稳定可靠地分流
  \item xxx (尽量使用量化的客观结果)
\end{itemize}
\end{onehalfspacing}

\datedsubsection{\textbf{\LaTeX\ 简历模板}}{2015 年5月 -- 至今}
\role{\LaTeX, Python}{个人项目}
\begin{onehalfspacing}
优雅的 \LaTeX\ 简历模板, https://github.com/billryan/resume
\begin{itemize}
  \item 容易定制和扩展
  \item 完善的 Unicode 字体支持,使用 \XeLaTeX\ 编译
  \item 支持 FontAwesome 4.5.0
\end{itemize}
\end{onehalfspacing}

% Reference Test
%\datedsubsection{\textbf{Paper Title\cite{zaharia2012resilient}}}{May. 2015}
%An xxx optimized for xxx\cite{verma2015large}
%\begin{itemize}
%  \item main contribution
%\end{itemize}

\section{\faCogs\ 技能}
% increase linespacing [parsep=0.5ex]
\begin{itemize}[parsep=0.5ex]
  \item 编程语言: 熟悉C/C++,了解go/python
  \item 平台: Linux
\end{itemize}

%% Reference
%\newpage
%\bibliographystyle{IEEETran}
%\bibliography{mycite}
\end{document}
